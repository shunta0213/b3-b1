\documentclass[../../../main]{subfiles}
\begin{document}

\section{方法}

\subsection{実験装置}
光学顕微鏡を平行になるように設置し、カメラを接続・映像をPCに出力できるようにした。
カメラには格子状のパターンを投影し、その画像をソフトウェアにて撮影できるようにした。
\footnote{
    ソフトウェアは定期的に画像を撮影できるものが好ましい。
    本実験では、 Motic Images Plus 2.3S を使用した。
}

\subsection{実験手順}
まず、あらかじめ粒径のわかっているポリエチレン溶液を



\end{document}
