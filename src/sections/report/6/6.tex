\subsection{課題6}

\subsubsection{}
マイクロメーターの1メモリの長さを測るのではなく、できる限り長く測る
なぜならば、1メモリの長さをはかろうとすると、測定誤差がそのまま
1メモリの誤差となる。
例えば、実際の値が\SI{10}{\micro\meter}で測定値が\SI{10.01}{\micro\meter}であるとすると、
誤差は\SI{0.01}{\micro\meter}である。
しかし、できる限り長いメモリ分を計測すればその分割ることができ誤差は小さくなる。
6メモリ分計測し、実際の値が\SI{60}{\micro\meter}で測定値が\SI{60.01}{\micro\meter}であるとする、
誤差は$\SI{0.01}{\micro\meter} / 6 = \SI{0.0017}{\micro\meter}$となる。

\subsubsection{}
マイクロメータの線と垂直に計測できるようなソフトウェアを使用する。
マイクロメータの線に対してソフトウェア上では直線を引き計測したが、
この直線とマイクロメータのなす角が\SI{90}{\degree}からずれていれば、誤差が生じる。
例えば、$\Delta \theta$だけずれていたとすると、実施の値を$l_r$、測定値を$l_m$とすると、
\begin{align*}
    l_r & = l_m \cos \Delta \theta
\end{align*}
となる。
$\Delta \theta=\SI{1}{\degree}$で$\cos\Delta\theta = 0.9998477$であるから、
誤差はそこまで大きくなるわけではないが、
できるだけ正確に測定するためには、マイクロメータの線と垂直に計測することが望ましい。
