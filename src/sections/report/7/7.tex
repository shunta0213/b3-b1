\subsection{課題7}

\subsubsection{熱雑音}
熱雑音とは、抵抗体内の電子の熱運動によって生じる雑音のことである。
1927年に発見され、ジョンソン-ナイキスト・ノイズとも呼ばれる。
熱雑音の電圧$V_n$は、
\begin{align*}
    V_n = \sqrt{4kTBR}
\end{align*}
と表される。
ここで、$k$はボルツマン定数、$T$は温度、$B$は帯域幅、$R$は抵抗値である。
一般に理想的な帯域幅を持つことはないため、熱雑音の計算は簡単ではないが、
熱雑音の振幅はガウス分布に従い、その振幅スペクトルは平坦な白色雑音である。
