\subsection{課題2}\label{subsec:report-2}

表\ref{tab:position}をもとに、
$x_m$の分散$\ev{(x_m-\ev{x_m})^2}=\ev{(\Delta x_m)^2}$、
$y_m$の分散$\ev{(y_m-\ev{y_m})^2}=\ev{(\Delta y_m)^2}$、
移動量$\sqrt{(\Delta x_m)^2 + (\Delta y_m)^2}$の分散、$\ev{(\Delta x_m)^2 + (\Delta y_m)^2}$を求めると、
次の表\ref{tab:position-variance}のようになる。

\begin{longtable}{ccc}
    \caption{粒子の座標の分散} \label{tab:position-variance}                                                                                                                                     \\
    \hline $\ev{(\Delta x_m)^2}$ /\SI{}{\micro\meter\squared} & $\ev{(\Delta y_m)^2}$ /\SI{}{\micro\meter\squared} & $\ev{(\Delta x_m)^2 + (\Delta y_m)^2}$ /\SI{}{\micro\meter\squared} \\ \hline
    \endfirsthead
    \hline $\ev{(\Delta x_m)^2}$ /\SI{}{\micro\meter\squared} & $\ev{(\Delta y_m)^2}$ /\SI{}{\micro\meter\squared} & $\ev{(\Delta x_m)^2 + (\Delta y_m)^2}$ /\SI{}{\micro\meter\squared} \\ \hline
    \endhead
    \hline
    \endfoot

    3.86                                                      & 2.52                                               & 6.39                                                                \\
\end{longtable}


ブラウン運動の力学的な考察から、
\begin{equation}\label{eq:brownian-motion}
    \ev{(\Delta x_m)^2} = \dfrac{kT}{3\pi a \eta} t = \dfrac{RT}{3 \pi a \eta N_A} t
\end{equation}
が成り立つ。
ここで、$k$はボルツマン定数、$T$は温度、$a$は粒子の半径、$\eta$は溶媒の粘度、$R$は気体定数、$N_A$はアボガドロ数である。
理科年表\cite{rika-nenpyo}によると、
室温は$T=\SI{299.35}{\kelvin}$、
粒子半径は$a=\SI{1.50}{\micro\meter}$、
気体定数は$R=\SI{8.3144598}{\joule\per\kelvin\per\mole}$
である。
$\eta$は水の粘度であるが、次の表\ref{tab:water-viscosity}のようになる。
指数関数でフィッティングすると、
次の\ref{eq:water-viscosity}式のようになりこれをもとに室温\SI{26.6}{\celsius}で水の粘度は\SI{0.84965}{\milli\pascal\second}であると考えられる。
\begin{equation}\label{eq:water-viscosity}
    \eta = 1.7599 \times \exp(-0.027793T) \si{\milli\pascal\second}
\end{equation}

\begin{longtable}{cc}
    \caption{水の温度と粘度の関係} \label{tab:water-viscosity}               \\
    \hline 温度 / \SI{}{\celsius} & 粘度 / \SI{}{\milli\pascal\second} \\ \hline
    \endfirsthead
    \hline 温度 / \SI{}{\celsius} & 粘度 / \SI{}{\milli\pascal\second} \\ \hline
    \endhead
    \hline
    \endfoot
    0                           & 1.7906                           \\
    5                           & 1.5185                           \\
    10                          & 1.3064                           \\
    15                          & 1.1378                           \\
    20                          & 1.0016                           \\
    25                          & 0.8899                           \\
    30                          & 0.7970                           \\
\end{longtable}


以上を踏まえて、アボガドロ定数、ボルツマン定数を計算すると以下のようになる。
\begin{align}
    k_{x_m} & = \dfrac{\ev{(\Delta x_m)^2} \cdot 3 \pi a \eta}{tT}                                                                                                            \nonumber          \\
            & = \dfrac{\SI{3.86}{\micro\meter\squared} \cdot 3 \pi \cdot \SI{1.50}{\micro\meter} \cdot \SI{0.84965}{\milli\pascal\second}}{\SI{5}{\second} \cdot \SI{299.35}{\kelvin}} \nonumber \\
            & = \SI{3.10E-23}{\joule\per\kelvin}                                                                                                                                                 \\
    k_{y_m} & = \SI{2.03E-23}{\joule\per\kelvin}
\end{align}
\begin{align}
    N_{x_m} & = \dfrac{R}{k_{x_m}} = \dfrac{\SI{8.3144598}{\joule\per\kelvin\per\mole}}{\SI{3.10E-23}{\joule\per\kelvin}} \nonumber \\
            & = \SI{2.68E23}{\per\mole}                                                                                             \\
    N_{y_m} & = \SI{4.09E23}{\per\mole}
\end{align}

なお、文献値は$N_A=\SI{6.02214076E23}{\per\mole}$、$k=\SI{1.38064852E-23}{\joule\per\kelvin}$である。
